\documentclass{article}
\usepackage{hyperref}
\usepackage{natbib}
\bibliographystyle{IEEEtranN}
\title{Analysing news articles impact \\ in stock price movements}
\author{Marcelo Grossi}
\begin{document}
\maketitle
\begin{abstract}
Stock market prediction has always been a topic of great interest for researchers. Numerous attempts to ``beat the market'' have been made without been able to consistently and accurately predict the movement of stock prices.
\par
In recent years, following the increase in computational processing capabilities, researchers have been studying the relationship between news articles and stock price movement \citep{Fu2008}\citep{Schumaker2009}. Most methods are based on sentiment analysis, where the news content is classified according to its potential of moving the underlying stock’s prices up (good news) or down (bad news). These methods however, do not take into account the financial instrument's context in determining the content of the news article, which greatly impairs the results of such studies.
\par
This work assumes the strong efficient market hypothesis which dictates that the stock's prices reflects all the information available (including historical prices, public news information and even insider information) \citep{fama1965behavior} and that everyone has some degree of access to the information. We will look at historical prices from different sources/indices and investigate the relationship between stock prices and news articles by identifying the important events in them and trying to match the news from that period with important price movements. This technique will allow us to identify for instance, which terminology or characteristic of news item has good or bad connotations to the price and how much impact can be attributed to it.
\par
As recent terrorist events have shown, some tend to be local with regard to stock market impact, others have more global ramifications.  This technique allow us to differentiate and localize the impact of a given new piece. Many other applications for this work can be thought of including, but not limited to a trade system that tries to anticipate price movements based on news repercussion and grouping of stocks that have similar news influencers but are not necessarily in the same market sector (discovery of non-obvious stock clustering).
\end{abstract}
\section{First Section}
\section{Second Section}
\section{Third Section}
\section{Conclusions}
\bibliography{practicum_marcelo_grossi}
\end{document}